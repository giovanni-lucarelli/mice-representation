\section{Conclusion}

In this project, we developed a biologically-informed preprocessing pipeline to feed visual stimuli to artificial neural networks that more closely resemble those perceived by mice. We trained AlexNet on the transformed ImageNet dataset and evaluated its neural similarity to biological responses recorded in the Allen Neuropixels dataset.

Our results demonstrate that the ecological data diet--that is, the preprocessing strategy--has a significant impact on neural predictivity compared to models trained without it. Notably, we observed a distinct and pronounced effect of the inference-time data relative to the training-time data. One possible explanation, considering the dot-product relationship between weights and input signals, is that the transformed stimuli themselves contribute more directly to neural predictivity than the weights optimized on such data. In other words, the mouse-like transformations applied at inference may be the primary drivers of alignment with biological neural activity. This finding raises an intriguing hypothesis: could inference-time transformations alone suffice to adapt a generic model such as AlexNet into an effective ``virtual brain'' for different species, by merely modifying its ``artificial retina''?

Two additional observations merit attention. First, our analysis revealed  an important interaction between the data diet and the network's learned representations. While all tested data diets improved the predictivity of pretrained models, they consistently decreased the predictivity of an untrained network with random weights. This indicates that the diet engages meaningfully with learned feature representions. Furthermore, Nayebi's diet shows stronger effects in the first convolutional layer, a possible explanation is that the interpolation and subsequent upscaling of the stimuli are alleviating the degradation that the global average pooling imposes by discarding much needed spatial structure for \texttt{conv1} to encode. 

These findings highlight the importance of stimulus realism in brain-model correspondence analysis. They also reveal that even simple manipulations of input distributions can substantially alter these measurements. Further analyses should explore these directions to better understand how ecological preprocessing shapes the neural alignment of artificial and biological systems.