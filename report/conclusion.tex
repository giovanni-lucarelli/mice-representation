\section{Discussion}

In this project, we developed a biologically-informed preprocessing pipeline to feed visual stimuli to artificial neural networks that more closely resemble those perceived by mice. We trained AlexNet on the transformed ImageNet dataset and evaluated its neural similarity to biological responses recorded in the Allen Neuropixels dataset.

Our results demonstrate that the ecological data diet--that is, the preprocessing strategy--has a significant impact on neural predictivity compared to models trained without it.  The diet has a negative impact on the first convolutional layer whereas, in the subsequent layers we always have an improvement (see \cref{fig:rsa-vsrandom}). For every visual area, the diet is particularly effective in \texttt{conv2}, \texttt{conv3}. Therefore we do not have evidence of a hierarchical structure in the mouse visual cortex, as already stated by Nayebi \textit{et al.}

Moreover, we found a clear distintion in the results of the neural predictivity when the diet is provided at training or inference time (see \cref{fig:inference}). While all tested data diets improved the predictivityof pretrained models, they consistently decreased the predictivity of untrained networks with random weights (see \cref{fig:random}). This indicates that the diet engages meaningfully with learned feature representions. In the case of trained models (both with diet and without) the inference time diet is decisive: the best model (except for the \texttt{conv1}) is given by the one trained with diet and with data at inference time, but the largest improvement in predictivity is the one observed when adding the diet at inference time with the model trained on non-preprocessed images, leading to a score almost always close to the one obtained by the model trained on preprocessed images (see \cref{fig:inference}). The improvement provided by the inference time diet on the model trained with the diet is marginal.

Lastly, our attempt to replicate the pseudo-diet proposed by Nayebi--performed in our analysis only at inference time-- showed that, in their work, they managed to find a very good proxy for an ecological diet by just rescaling the images to a lower resolution, averaging out some infomation. Although in the convolutional layers from 2 to 4 our diet still provides the best predictivity, in \texttt{conv1} the pseudo-diet has the best results (see \cref{fig:nayebi}). In the pseudo-diet we perform a rescaling --without losing spatial information-- whereas in our diet we perform a random resize crop, possibly loosing spatial information. Since the predictivity score is computed through Pearson correlation, we may loose sources of explainability of the variation in doing such transformation. This may lead to a worsening of the performance on \texttt{conv1} layer, since it strongly relies on the input's spatial structure by contruction.


% one open question, possible closure

Our initial research question was to find a model for the mouse visual system by taking into account the animal's visual experience. Hence our reference model is the one with diet both at training and inference time, \textit{i.e.}, a model that has always seen and will always see mouse-like images. The astonishing observation is that the main effect comes at inference time, even in a model that has learned its weights on regular images, with ``perfect acuity". This suggests that the mouse-like transformations applied at inference might be the primary drivers of alignment with biological neural activity. 

This finding raises an intriguing hypothesis: could inference-time transformations alone suffice to adapt a generic model such as AlexNet into an effective ``virtual brain'' for different species, by merely modifying its ``artificial retina''? 

Together these findings highlight the importance of stimulus realism in brain-model correspondence analysis. They also reveal that even simple manipulations of input distributions can substantially alter these measurements. Further analyses should explore these directions to better understand how ecological preprocessing shapes the neural alignment of artificial and biological systems.