\vspace{2em}

\section{Introduction}

Studies of the mouse visual system have identified a range of cortical areas that support diverse visual behaviors, including stimulus-reward associations, goal-directed navigation, and object-centered discrimination. Despite extensive investigation, a comprehensive understanding of the mouse visual cortex and its functional organization remains incomplete.

Nayebi \textit{et al.}~\cite{nayebi} demonstrated that a shallow neural network architecture trained with a self-supervised objective and low-resolution visual inputs provides an optimal model of the mouse visual cortex. Their findings suggest that a lightweight, general-purpose visual system can effectively account for mouse visual representations. Acknowledging the inherently low visual acuity of mice, Nayebi \textit{et al.} achieved improved neural predictivity by training their network on lower-resolution images. They used image resolution as a proxy for mouse visual acuity, rather than employing a biologically informed image preprocessing pipeline to approximate this property. They reported optimal neural predictivity using $64\times64$ pixel images and proposed that future work should incorporate more realistic, neurophysiologicallys informed preprocessing approaches.

The goal of this project is to extend Nayebi's analysis by developing and evaluating such preprocessing pipeline. Specifically, this study aims to investigate whether biologically informed visual transformations improve neural predictivity. Our findings indicate that the proposed tailored data diet yields higher neural predictivity compared to a traditional ImageNet-pretrained model. Moreover, a post-hoc analysis revealed that applying the data diet at inference time has a substantial impact on models trained without it. Together, these findings highlight the importance of stimulus realism and preprocessing in modeling mouse visual systems. They also suggest that even simple manipulations of the input distribution can yield large changes in neural predictivity, highliting a potential confound in Nayebi's work, where such effects were not controlled for.

